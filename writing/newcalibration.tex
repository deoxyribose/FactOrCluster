\documentclass{article}

%symbols
\usepackage{mathtools}
\usepackage{amssymb}
\usepackage{physics}

%typesetting
\usepackage{bm} %bold math symbols
\usepackage{bbm}
\usepackage{dsfont}
\usepackage{stackrel}
%Symbols
\newcommand{\constant}{\operatorname{const.}}
\newcommand{\bigoh}{\mathcal{O}}

%Data Science
\newcommand{\data}{\mathcal{D}}
\newcommand{\model}{\mathcal{M}}
\newcommand{\tol}{\epsilon}
\DeclareMathOperator*{\argmin}{{argmin}}
\DeclareMathOperator*{\argmax}{{argmax}}

%Special functions
\DeclareMathOperator{\IndicatorOp}{\mathbbm{1}}
\newcommand{\indicator}[1]{\IndicatorOp\!\left[#1\right]}
\newcommand{\inv}{^{-1}}
\newcommand{\inprod}[2]{\left\langle#1,#2\right\rangle}
\newcommand{\inprodv}[2]{\transpose{#1}#2}
\newcommand{\outerprod}[2]{#1\transpose{#2}}
\newcommand{\const}{\text{const.}}
\newcommand{\evalat}[2]{\left.#1\right\rvert_{#2}}

%Vectors/Matrices
\newcommand{\diag}{\operatorname{Diag}}
\newcommand{\T}{^\top}
\newcommand{\transpose}[1]{#1^\top}
\newcommand{\Tinv}{^{-\top}}
\DeclareMathOperator{\VecOp}{vec}
\newcommand{\vectorize}[1]{\VecOp\!\left(#1\right)}


%Matrices and Vectors
\newcommand{\x}{\bm{x}}
\newcommand{\X}{\bm{X}}
\newcommand{\y}{\bm{y}}
\newcommand{\Y}{\bm{Y}}
\newcommand{\z}{\bm{z}}
\newcommand{\Z}{\bm{Z}}
\newcommand{\w}{\bm{w}}
\newcommand{\W}{\bm{W}}
\newcommand{\vecu}{\bm{u}}
\newcommand{\U}{\bm{U}}
\newcommand{\vecv}{\bm{v}}


%Probability
\DeclareMathOperator{\Eop}{\mathbb{E}}
\newcommand{\E}[1]{\Eop\!{\left[#1\right]}}
\newcommand{\Ein}[2]{\Eop_{#1}\!\left[#2\right]}
\DeclareMathOperator{\Varop}{\mathbb{V}}
\newcommand{\Varv}[1]{\Varop\!{\left[#1\right]}}
\DeclareMathOperator{\Varfun}{\mathrm{Var}}
\newcommand{\Var}[1]{\Varfun\!{\left(#1\right)}}
\newcommand{\Varin}[2]{\Varfun_{#1}\!{\left(#2\right)}}
\DeclareMathOperator{\Covop}{\mathrm{Cov}}
\newcommand{\Cov}[1]{\Covop\!{\left(#1\right)}}
\DeclareMathOperator{\Corrop}{\mathrm{Corr}}
\newcommand{\Corr}[1]{\Corrop\!{\left(#1\right)}}
\newcommand{\sampCov}[1]{\estim{\Covop}\!{\left(#1\right)}}
\newcommand{\Covin}[2]{\Covop_{#1}\!{\left(#2\right)}}

\newcommand{\KL}[2]{\operatorname{KL}({#1}\|{#2})}
\newcommand{\entropy}{\mathcal{H}}
\newcommand{\prob}{p}
\newcommand{\cond}{|}
\newcommand{\indep}{\perp\!\!\!\perp}
\newcommand{\score}{\mathcal{V}}
\newcommand{\fisher}{\mathcal{I}}
\newcommand{\partitionfunc}{\mathcal{Z}}
\newcommand{\disteq}{\stackrel{d}{=}}
\newcommand\independent{\protect\mathpalette{\protect\independenT}{\perp}}
\def\independenT#1#2{\mathrel{\rlap{$#1#2$}\mkern2mu{#1#2}}}

%distributions
\newcommand{\gaussrnd}{\operatorname{\mathcal{N}}}
\newcommand{\betarnd}{\operatorname{Beta}}
\newcommand{\berrnd}{\operatorname{Ber}}
\newcommand{\multrnd}{\operatorname{Mult}}
\newcommand{\dirrnd}{\operatorname{Dir}}
\newcommand{\igrnd}{\operatorname{IG}}
\newcommand{\nigrnd}{\operatorname{NIG}}
\newcommand{\unifrnd}{\operatorname{U}}

%graphs
\newcommand{\graph}{G}
\newcommand{\edges}{E}
\newcommand{\vertices}{V}
\newcommand{\adjmat}{A}
\newcommand{\valence}{v}

%Sets
\newcommand{\reals}{\mathbb{R}}
\newcommand{\complex}{\mathbb{C}}
\newcommand{\naturals}{\mathbb{N}}
\newcommand{\integers}{\mathbb{Z}}
\newcommand{\domain}{\Omega}
\newcommand{\paramspace}{\Theta}
\newcommand{\except}[1]{\backslash#1} %every element except
\newcommand{\intersection}{\cap}
\newcommand{\itset}[3]{\lbrace#1_{#2}\rbrace_{#2=1}^{#3}}

\newcommand{\mix}{\bm{A}}
\newcommand{\source}{s}
\newcommand{\sourcevec}{\bm{\source}}

\newcommand{\obs}{y}
\newcommand{\obsvec}{\bm{\obs}}
\newcommand{\mean}{\mu}
\newcommand{\meanvec}{\bm{\mean}}
\newcommand{\scale}{\sigma}
\newcommand{\scalesq}{\scale^2}
\newcommand{\sourcecov}{\bm{\Sigma}}
\newcommand{\weight}{w}
\newcommand{\weightvec}{\bm{\weight}}
\newcommand{\compweight}{\bar{\weight}}
\newcommand{\compweightvec}{\bar{\bm{\weight}}}
\newcommand{\assign}{z}
\newcommand{\assignvec}{\bm{\assign}}
\newcommand{\compmean}{\bar{\mean}}
\newcommand{\compmeanvec}{\bar{\meanvec}}
\newcommand{\compcov}{\bar{\bm{\Sigma}}}
\newcommand{\compscalesq}{\bar{{\scale}}^2}
\newcommand{\noisecov}{\bm{\Lambda}}
\newcommand{\noisescalesq}{\lambda^2}
\newcommand{\noisevec}{\bm{e}}

\begin{document}
Gaussian mixtures, factor models, independent factor analysis, and mixtures of factor analyzers are all different restrictive types of Gaussian mixtures with density
\begin{equation}
p(\obsvec_n)=\sum_{q=1}^{Q}\compweight_{q}\gaussrnd(\obsvec_n|\compmeanvec_q,\compcov_q)
\end{equation} 

The mean and second moment of a Gaussian mixture is given by
\begin{align}
\E{\obsvec_n}=&\sum_{q=1}^{Q}\compweight_q\compmeanvec_q\\
\E{\obsvec_n\obsvec_n\T}=&\sum_{q=1}^{Q}\compweight_q(\compcov_q + \compmeanvec_q\compmeanvec_q\T)
\end{align}

The covariance then follows from the standard definition,
\begin{equation}
\Cov{\obsvec_n}=\E{\obsvec_n\obsvec_n\T}-\E{\obsvec_n}\E{\obsvec_n}\T.
\end{equation}

In the above we implicitly conditioned on a number of hyperparameters. If we assume that they are all independent and that $\E{\compweight_q}=1/Q$ and $\E{\compmeanvec_q}=\bm{0}$ and $\E{\compmeanvec_q\compmeanvec_q\T}=\noisecov_{\compmean}$, then
\begin{align}
\E{\obsvec_n}=&\bm{0}\\
\label{eq:margcov}
\E{\obsvec_n\obsvec_n\T}=\Cov{\obsvec_n}=&\noisecov_{\compmean}+\frac{1}{Q}\sum_{q=1}^{Q}\E{\compcov_q}
\end{align} 
We can now further assume that the covariance matrix has random low-rank structure of the form,
\begin{equation}
\compcov_q=\mix \sourcecov_q\mix\T+\noisecov.
\end{equation}
which leads to the intractable second moment,
\begin{align}
\E{\obsvec_n\obsvec_n\T}=\Cov{\obsvec_n}=&\noisecov_{\compmean}+\noisecov+\frac{1}{Q}\sum_{q=1}^{Q}\E{\mix\sourcecov_q\mix\T}.
\end{align}
To render it tractable, we can calculate the trace of the matrix, which is simply
\begin{equation}
\trace(\Cov{\obsvec_n})=\trace\left(\mix\T\mix\left(\sum_{q=1}^{Q}\compweight_q\sourcecov_q\right)\right)+\trace(\E{\noisecov})+\trace(\noisecov_{\compmean}).
\end{equation}
The total covariance is also amenable to marginalization over the hyperparameters yielding a, 
\begin{equation}
\trace(\Cov{\obsvec_n})=\trace\left(\E{\mix\T\mix}\left(\frac{1}{Q}\sum_{q=1}^{Q}\E{\sourcecov_q}\right)\right)+\trace(\E{\noisecov})+\trace(\noisecov_{\compmean}).
\end{equation}
If we assume that $\mix$ is a standard Gaussian ensemble with $[\mix]_{ij}\sim \gaussrnd(0,1)$ then $\mix\T\mix$ follows a Wishart distribution with mean $d\bm{I}$ where $d$ is the dimensionality of $\obsvec_n$. The total covariance then simplifies to 
\begin{equation}
\label{eq:totalcov}
\trace(\Cov{\obsvec_n})=\frac{d}{Q}\sum_{q=1}^{Q}\trace(\E{\sourcecov_q})+\trace(\E{\noisecov})+\trace(\noisecov_{\compmean}).
\end{equation}
If we assume that $\noisecov_{\compmean}=\scalesq_0\bm{I}$ and $\E{\noisecov}=m_{\noisescalesq}\bm{I}$
\begin{equation}
\label{eq:totalcov_diag}
\trace(\Cov{\obsvec_n})=\frac{d}{Q}\sum_{q=1}^{Q}\trace(\E{\sourcecov_q})+d\sigma_0^2+dm_{\noisescalesq}.
\end{equation}

\section{Centered Independent Factor Analysis}
For a centered independent factor analysis model, there is no mean component and $\compcov_q$ is diagonal and has inverse Gamma distributions on its diagonal elements. Take the distribution to have mean ${m}_{\scalesq}$. The resulting total covariance is
\begin{equation}
\trace(\Cov{\obsvec_n})=dKm_{\scalesq}+dm_{\noisescalesq}.
\end{equation}
If we set the total covariance to be equal to $d$ and let $m_{\noisescalesq}$ range freely, this imposes the constraint
\begin{equation}
m_{\scalesq}=\frac{1-m_{\noisescalesq}}{K},
\end{equation}
essentially distributing the remaining variance evenly across the source dimensions. To tune the variances, see section \ref{sec:invgamma}.

To encourage non-Gaussian patterns, we can impose a different inverse Gamma prior on one (or more of the mixture components). If we let the $q=1$ cluster have prior mean $\rho\scalesq$ instead, with $\rho\in(0,1]$, then the total covariance becomes
\begin{equation}
\trace(\Cov{\obsvec_n})=dKm_{\scalesq}-dKC^{-1}(1-\rho)m_{\scalesq}+dm_{\noisescalesq}.
\end{equation}
with resulting solution,
\begin{equation}
m_{\scalesq}=\frac{1-m_{\noisescalesq}}{K(1-C^{-1}(1-\rho))}.
\end{equation}

\section{Projected Mixture}
We define projected mixtures to be mixtures on a low-rank space that are then projected up into the observation space using a factor matrix $\mix$ as with a factor model. This corresponds to the above model structure except for the fact that we need to take the mean to be low-rank $\compmeanvec_q=\mix {\compmeanvec}'_q$ where $\compmeanvec'_q\sim\gaussrnd(\bm{0},\alpha_0\bm{I})$ so that
\begin{equation}
\trace(\Cov{\compmeanvec_q})=\scalesq_0\trace(\E{\mix\T\mix})=dK\scalesq_0.
\end{equation}
If we simultaneously let $\compcov_q$ follow an inverse Wishart which has mean 
\begin{equation}
\E{\compcov_q}=\frac{\rho}{\nu - K - 1}\bm{I}
\end{equation}
for scale matrix $\rho \bm{I}$ and degrees of freedom $\nu$ we get the total covariance,
\begin{equation}
\trace(\Cov{\obsvec_n})=dK\scalesq_0+\frac{dK\alpha}{\nu-K-1}+dm_{\noisescalesq}.
\end{equation}  
To balance the terms, we match the variance contribution of the mean and the components, 
\begin{equation}
\scalesq_0=\frac{\rho}{\nu-K-1},
\end{equation}
which leads to the following constraint on $\alpha$ and $\nu$,
\begin{equation}
\frac{\rho}{\nu-K-1}=\frac{1-m_{\noisescalesq}}{2K}
\end{equation}
We note that $[\compcov_q]_{ii}$ is distributed as an inverse gamma,
\begin{equation}
[\compcov_q]_{ii}\sim \operatorname{IG}\left(\frac{\nu-K+1}{2},\frac{\rho}{2}\right)
\end{equation}
with mean $\frac{\rho}{\nu-K-1}$, so we propose tuning an inverse gamma $\operatorname{IG}(\alpha,\beta)$ to have mean $\frac{1-m_{\noisescalesq}}{2K}$ and appropriate variance following section \ref{sec:invgamma}, and then setting
\begin{equation}
\nu=K+2\alpha-1,\quad \rho=2\beta.
\end{equation}



\section{Tuning the Inverse Gamma Variances}
\label{sec:invgamma}
For an inverse Gamma distribution $\operatorname{InvGamma}(\alpha,\beta)$ the mean and variance does not exist unless $\alpha>2$, so we consider a random variable $X\sim\operatorname{InvGamma}(2+\alpha,\beta)$. $X$ then has mean and variance equal to 
\begin{align}
\E{X}=&\frac{\beta}{\alpha+1}\\
\Var{X}=&\frac{\beta^2}{\alpha(\alpha+1)^2}
\end{align}
If we set $\E{X}=m$ and $\Var{X}=v$, then we can isolate the alpha and beta parameters in terms of the mean and variance as,
\begin{align}
\beta=m\left(1+\frac{m^2}{v}\right),\quad 
\alpha=\frac{m^2}{v}
\end{align}



Employing Markov's inequality, we can bound the tail probability using both the mean and the variance as 
\begin{equation}
\mathbb{P}[X\geq a]\leq \frac{\E{X}}{a},\quad \mathbb{P}[X\geq a]\leq \frac{\Var{X}}{(a-\E{X})^2}.
\end{equation}
Assume that the mean is set equal to $m$. Then if we want less than $t$ mass in the tails, we can select the variance to be
\begin{equation}
\Var{X}=t(a-m)^2
\end{equation}

For a noise variable, we are likely to want a low mean variance like $m=10^{-1}$, and we likely want to contain (at least) $1-t=0.95$ of the probability mass to $[0,1]$ requiring $a=1$. This results in a proposed variance of 
\begin{equation}
\Var{X}=\frac{1}{20}(1-10^{-1})^2=0.0405
\end{equation} 
this roughly holds if $\beta=0.1247$ and $\alpha=0.2469$ (or $\alpha=2.2469$ in the original parameterization).

\subsection{Tuning the Projected Mixture}
For the projected mixture, we need to fix four effective parameters. We make the assumption that,
\begin{equation}
K\alpha_0=\frac{K\alpha}{\nu-K-1}
\end{equation}
which corresponds to matching the total covariance contribution of the component means and the observations. Assuming the total covariance is $d$ (corresponding to an identity covariance matrix) then we have,
\begin{equation}
1-m_{\noisescalesq}=\frac{2K\alpha}{\nu-K-1}.
\end{equation}
Using this, we can isolate $\alpha$,
\begin{equation}
\alpha=(\nu-K-1)\frac{1-m_{\noisescalesq}}{2K}.
\end{equation}
For the inverse Wishart the diagonal elements $\compcov_{ii}$ have variance,
\begin{equation}
\Var{\compcov_{ii}}=\frac{2\alpha^2}{(\nu-K-1)^2(\nu-K-3)}.
\end{equation}
If we fix the variance, we can insert the above value for $\alpha$ and isolate $\nu$ as
\begin{equation}
\nu=K+3+2\frac{(\frac{1-m_{\noisescalesq}}{2K})^2}{\Var{\compcov_{ii}}}
\end{equation}
Adding it back in we get the following formula for $\alpha$,
\begin{equation}
\alpha=\left(1+\frac{\left(\frac{1-m_{\noisescalesq}}{2K}\right)^2}{\Var{\compcov_{ii}}}\right)\frac{1-m_{\noisescalesq}}{K}
\end{equation}

With Wishart priors, we start with
\begin{equation}
\alpha=\frac{1-m_{\noisescalesq}}{2K\nu},
\end{equation}
and using the variance expression,
\begin{equation}
\Var{\compcov_{ii}}=2\nu\alpha^2,
\end{equation}
we can isolate $\nu$,
\begin{equation}
\nu=\frac{(1-m_{\noisescalesq})^2}{2K^2\Var{\compcov_{ii}}}
\end{equation}
and then find $\alpha$,
\begin{equation}
\alpha=\frac{K\Var{\compcov_{ii}}}{1-m_{\noisescalesq}}.
\end{equation}

\subsubsection{Alternative Calibration of the Inverse Wishart}
Alternatively we can use that if $\compcov\sim \mathcal{W}^{-1}(\alpha\bm{I},\nu)$ then,
\begin{equation}
\compcov_{ii}\sim \Gamma^{-1}\left(\frac{\alpha}{2},\frac{\nu-K-1}{2}\right)
\end{equation}
which means we can calibrate the Wishart as an Inverse Gamma. We should then set 
\begin{equation}
\alpha_0=\frac{\alpha}{\nu-K-1}
\end{equation}
to balance 
\subsection{Low-rank Gaussian density}
\newcommand{\excov}{\bm{\Sigma}}
The Gaussian density function is given by,
\begin{equation}
\frac{1}{\sqrt{(2\pi)^d}|\excov|}\exp(-\frac{1}{2}(\x-\meanvec)\T\excov^{-1}(\x-\meanvec))
\end{equation}
If $\excov$ is low-rank,
\begin{equation}
\excov=\mix\bm{S}\mix\T+\bm{D}
\end{equation}
then we can employ tricks to calculate the inverse more efficiently. The Woodbury matrix identity yields,
\begin{equation}
(\mix\bm{S}\mix\T+\bm{D})^{-1}=\bm{D}^{-1}-\bm{D}^{-1}\mix(\bm{S}^{-1}+\mix\T\bm{D}^{-1}\mix)^{-1}\mix\T\bm{D}^{-1},
\end{equation}
which can be simplified further if the Cholesky factorization $\bm{L}\bm{L}\T=\bm{S}$ is known, as we can pull the factors outside,
\begin{equation}
(\mix\bm{S}\mix\T+\bm{D})^{-1}=\bm{D}^{-1}-\bm{D}^{-1}\mix\bm{L}(\bm{I}+\bm{L}\T\mix\T\bm{D}^{-1}\mix\bm{L})^{-1}\bm{L}\T\mix\T\bm{D}^{-1}.
\end{equation}
If we define $\bm{B}=\bm{D}^{-1}\mix\bm{L}$ then this can be written fairly succinctly as,
\begin{equation}
(\mix\bm{S}\mix\T+\bm{D})^{-1}=\bm{D}^{-1}-\bm{B}(\bm{I}+\bm{B}\T\bm{D}\bm{B})^{-1}\bm{B}\T.
\end{equation}
which only requires inversion of diagonal matrices and matrices of the same shape as $\bm{S}$, which can be smaller than $\excov$ by design.

We can similarly apply the matrix determinant lemma to calculate the determinant as
\begin{equation}
|\mix\bm{S}\mix\T+\bm{D}|=|\bm{D}||\bm{I}+\bm{L}\T\mix\T\bm{D}^{-1}\mix\bm{L}|=|\bm{D}||\bm{I}+\bm{B}\T\bm{D}\bm{B}|
\end{equation}

\end{document}
