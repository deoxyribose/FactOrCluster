\documentclass{article}

%symbols
\usepackage{mathtools}
\usepackage{amssymb}
\usepackage{physics}

%typesetting
\usepackage{bm} %bold math symbols
\usepackage{bbm}
\usepackage{dsfont}
\usepackage{stackrel}
%Symbols
\newcommand{\constant}{\operatorname{const.}}
\newcommand{\bigoh}{\mathcal{O}}

%Data Science
\newcommand{\data}{\mathcal{D}}
\newcommand{\model}{\mathcal{M}}
\newcommand{\tol}{\epsilon}
\DeclareMathOperator*{\argmin}{{argmin}}
\DeclareMathOperator*{\argmax}{{argmax}}

%Special functions
\DeclareMathOperator{\IndicatorOp}{\mathbbm{1}}
\newcommand{\indicator}[1]{\IndicatorOp\!\left[#1\right]}
\newcommand{\inv}{^{-1}}
\newcommand{\inprod}[2]{\left\langle#1,#2\right\rangle}
\newcommand{\inprodv}[2]{\transpose{#1}#2}
\newcommand{\outerprod}[2]{#1\transpose{#2}}
\newcommand{\const}{\text{const.}}
\newcommand{\evalat}[2]{\left.#1\right\rvert_{#2}}

%Vectors/Matrices
\newcommand{\diag}{\operatorname{Diag}}
\newcommand{\T}{^\top}
\newcommand{\transpose}[1]{#1^\top}
\newcommand{\Tinv}{^{-\top}}
\DeclareMathOperator{\VecOp}{vec}
\newcommand{\vectorize}[1]{\VecOp\!\left(#1\right)}


%Matrices and Vectors
\newcommand{\x}{\bm{x}}
\newcommand{\X}{\bm{X}}
\newcommand{\y}{\bm{y}}
\newcommand{\Y}{\bm{Y}}
\newcommand{\z}{\bm{z}}
\newcommand{\Z}{\bm{Z}}
\newcommand{\w}{\bm{w}}
\newcommand{\W}{\bm{W}}
\newcommand{\vecu}{\bm{u}}
\newcommand{\U}{\bm{U}}
\newcommand{\vecv}{\bm{v}}


%Probability
\DeclareMathOperator{\Eop}{\mathbb{E}}
\newcommand{\E}[1]{\Eop\!{\left[#1\right]}}
\newcommand{\Ein}[2]{\Eop_{#1}\!\left[#2\right]}
\DeclareMathOperator{\Varop}{\mathbb{V}}
\newcommand{\Varv}[1]{\Varop\!{\left[#1\right]}}
\DeclareMathOperator{\Varfun}{\mathrm{Var}}
\newcommand{\Var}[1]{\Varfun\!{\left(#1\right)}}
\newcommand{\Varin}[2]{\Varfun_{#1}\!{\left(#2\right)}}
\DeclareMathOperator{\Covop}{\mathrm{Cov}}
\newcommand{\Cov}[1]{\Covop\!{\left(#1\right)}}
\DeclareMathOperator{\Corrop}{\mathrm{Corr}}
\newcommand{\Corr}[1]{\Corrop\!{\left(#1\right)}}
\newcommand{\sampCov}[1]{\estim{\Covop}\!{\left(#1\right)}}
\newcommand{\Covin}[2]{\Covop_{#1}\!{\left(#2\right)}}

\newcommand{\KL}[2]{\operatorname{KL}({#1}\|{#2})}
\newcommand{\entropy}{\mathcal{H}}
\newcommand{\prob}{p}
\newcommand{\cond}{|}
\newcommand{\indep}{\perp\!\!\!\perp}
\newcommand{\score}{\mathcal{V}}
\newcommand{\fisher}{\mathcal{I}}
\newcommand{\partitionfunc}{\mathcal{Z}}
\newcommand{\disteq}{\stackrel{d}{=}}
\newcommand\independent{\protect\mathpalette{\protect\independenT}{\perp}}
\def\independenT#1#2{\mathrel{\rlap{$#1#2$}\mkern2mu{#1#2}}}

%distributions
\newcommand{\gaussrnd}{\operatorname{\mathcal{N}}}
\newcommand{\betarnd}{\operatorname{Beta}}
\newcommand{\berrnd}{\operatorname{Ber}}
\newcommand{\multrnd}{\operatorname{Mult}}
\newcommand{\dirrnd}{\operatorname{Dir}}
\newcommand{\igrnd}{\operatorname{IG}}
\newcommand{\nigrnd}{\operatorname{NIG}}
\newcommand{\unifrnd}{\operatorname{U}}

%graphs
\newcommand{\graph}{G}
\newcommand{\edges}{E}
\newcommand{\vertices}{V}
\newcommand{\adjmat}{A}
\newcommand{\valence}{v}

%Sets
\newcommand{\reals}{\mathbb{R}}
\newcommand{\complex}{\mathbb{C}}
\newcommand{\naturals}{\mathbb{N}}
\newcommand{\integers}{\mathbb{Z}}
\newcommand{\domain}{\Omega}
\newcommand{\paramspace}{\Theta}
\newcommand{\except}[1]{\backslash#1} %every element except
\newcommand{\intersection}{\cap}
\newcommand{\itset}[3]{\lbrace#1_{#2}\rbrace_{#2=1}^{#3}}

\newcommand{\mix}{\bm{A}}
\newcommand{\source}{s}
\newcommand{\sourcevec}{\bm{\source}}
\newcommand{\sourcecov}{\Sigma}
\newcommand{\obs}{y}
\newcommand{\obsvec}{\bm{\obs}}
\newcommand{\mean}{\mu}
\newcommand{\meanvec}{\bm{\mean}}
\newcommand{\scale}{\sigma}
\newcommand{\scalesq}{\scale^2}
\newcommand{\weight}{w}
\newcommand{\weightvec}{\bm{\weight}}
\newcommand{\compweight}{\bar{\weight}}
\newcommand{\compweightvec}{\bar{\bm{\weight}}}
\newcommand{\assign}{z}
\newcommand{\assignvec}{\bm{\assign}}
\newcommand{\compmean}{\bar{\mean}}
\newcommand{\compmeanvec}{\bar{\meanvec}}
\newcommand{\compcov}{\bar{\bm{\Sigma}}}
\newcommand{\compscalesq}{\bar{{\scale}}^2}
\newcommand{\noisecov}{\bm{\Lambda}}
\newcommand{\noisescalesq}{\lambda^2}

\begin{document}
The Projected mixture of Gaussians (proj-MoG) model assumes a standard generative factor model for the observations $\bm{y}$ with $K$ factors,  
\begin{align*}
\obsvec_n\sim \gaussrnd(\mix\sourcevec_n,\noisecov),
\end{align*}
but uses a mixture of Gaussians with $Q$ clusters for the sources,
\begin{align*}
\sourcevec_{n}\sim \sum_{q=1}^{Q}\compweight_{q}\gaussrnd(\compmeanvec_{q},\compcov_{q}),
\end{align*}

As with any mixture, we can express the mixture density using auxiliary indicators, where $\assign_{nq}=1$ if observation $n$ was drawn from component $q$, and $0$ otherwise.
\begin{equation}
\sourcevec_n\cond \assignvec_n\sim\prod_{q=1}^Q\gaussrnd(\compmeanvec_{q},\compcov_q)^{\assign_{nq}},\quad \assignvec_n\sim \multrnd(\compweightvec).
\end{equation}

with $N$ observations, the full log-joint has terms
\begin{align*}
\ln p(\obsvec|\sourcevec)&=\sum_{n=1}^{N}\left(-\frac{1}{2}(\obsvec_n-\mix\sourcevec_n)\T\noisecov^{-1}(\obsvec_n-\mix\sourcevec_n)-\frac{1}{2}\ln|\noisecov|-\frac{D}{2}\ln 2\pi)\right)\\
\ln p(\sourcevec|\assignvec)&=\sum_{n=1}^{N}\sum_{q=1}^{Q}\assign_{nq}\left(-\frac{1}{2}(\sourcevec_n-\compmeanvec_q)\T\compcov^{-1}_q(\sourcevec_n-\compmeanvec_q)-\frac{1}{2}\ln|\compcov_q|-\frac{D}{2}\ln 2\pi)\right)\\
\ln p(\assignvec)&=\sum_{n=1}^{N}\sum_{q=1}^{Q}\assign_{nq}\ln \compweight_{q}
\end{align*}
To compute gradients in $\ln p(\obsvec)$, we need to be able to integrate $\sourcevec$ and $\assignvec$ over $p(\sourcevec,\assignvec|\obsvec)=p(\sourcevec|\assignvec,\obsvec)p(\assignvec|\obsvec)$. First we have a standard Gaussian posterior,
\begin{equation}
p(\sourcevec_n|\assign_{nq}=1,\obsvec_n)=\gaussrnd(\bm{V}_q\left(\mix\T\noisecov^{-1}\obsvec_n + \compcov_q^{-1}\compmeanvec_q\right),\bm{V}_q),\quad \bm{V}_q=(\mix\T\noisecov^{-1}\mix+\compcov^{-1}_q)^{-1}
\end{equation}
and then using a standard responsibility argument, we can find the posterior of the assignment indicators $\assignvec$ as
\begin{equation}
p(\assignvec_{nq}=1|\obsvec_n)\propto \compweight_{q}p(\obsvec_n|\assign_{nq}=1)=\compweight_{q}\gaussrnd(\obsvec_n|\mix\compmeanvec_q,\noisecov+\mix\compcov_q\mix\T).
\end{equation}
\subsection{Mixture weights}
Since $\sum_{q=1}^{Q}\compweight_{q}=1$, we have to add Lagrangians, and due to the positivity constraint we reparameterize as $v_{q}=\ln\compweight_{q}$. The gradient is then  
\begin{equation}
\nabla_{v_{q}}\left(\ln p(\obsvec,\sourcevec,\assignvec) +\sum_{\ell=1}^{K}\rho_\ell(1-\sum_{d=1}^{C}e^{v_{\ell d}})\right)=\sum_{n=1}^N\sum_{q=1}^Q\assign_{nq}-\rho_ke^{v_{q}}
\end{equation}
Taking the expectation and solving we find,
\begin{equation}
\compweight_{q}\propto\sum_{n=1}^{N}\E{\assign_{nq}}
\end{equation}
If we add a Dirichlet prior $\compweightvec\sim\dirrnd(\bm{\alpha})$, then it contributes with the terms 
\begin{equation}
\nabla_{v_{q}}\ln p(\weightvec)=\alpha_{q}-1
\end{equation}
which changes the analytical solution to
\begin{equation}
\compweight_{q}\propto (\alpha_{q}-1)+\sum_{n=1}^{N}\E{\assign_{nq}}
\end{equation}
\subsection{Mixture Mean}
\begin{equation}
\nabla_{\compmeanvec_q}\ln p(\obsvec,\sourcevec,\assignvec)=-\frac{1}{2}\sum_{n=1}^{N}\assign_{nq}\left(\compcov_q^{-1}(\sourcevec_n-\compmeanvec_q)\right)
\end{equation}
taking the expectation, setting to zero, and isolating,
\begin{equation}
\compmeanvec_q=\frac{1}{\sum_{n=1}^{N}\E{\assign_{nq}}}\sum_{n=1}^{N}\E{\assign_{nq}}\E{\sourcevec_n|\assign_{nq}=1}
\end{equation}
or we can introduce a prior in the form of a normal, \begin{equation}
\nabla_{\compmeanvec_q}\ln\gaussrnd(\compmeanvec_q|\compmeanvec_0,\tau\bm{I})=-\frac{1}{2\tau}\nabla_{\compmeanvec_q}\left(\compmeanvec_q-\compmeanvec_0\right)\T\left(\compmeanvec_q-\compmeanvec_0\right)=-\frac{1}{\tau}(\compmeanvec_q-\compmeanvec_0)
\end{equation}
which we can add in before solving,
\begin{equation}
\compmeanvec_q=\left(\frac{1}{\tau}\compcov_q+\left(\sum_{n=1}^{N}\E{\assign_{nq}}\right)\bm{I}\right)^{-1}\left(\frac{1}{\tau}\compcov_q\compmeanvec_0+\sum_{n=1}^{N}\E{\assign_{nq}}\E{\sourcevec_n|\assign_{nq}=1}\right)
\end{equation}
If we use the conjugate prior $\gaussrnd(\compmeanvec_q|\compmeanvec_0,\tau\compcov_q)$ instead, this simplifies to an expression that does not involve matrix inverses,
\begin{equation}
\compmeanvec_q=\frac{1}{\frac{1}{\tau}+\sum_{n=1}^{N}\E{\assign_{nq}}}\left(\frac{1}{\tau}\compmeanvec_0+\sum_{n=1}^{N}\E{\assign_{nq}}\E{\sourcevec_n|\assign_{nq}=1}\right)
\end{equation}




\subsection{Mixture Variance}
Taking the gradient in $\compcov_q$, we get
\begin{equation}
\nabla_{\compcov_q}\ln p(\obsvec,\sourcevec,\assignvec)=\frac{1}{2}\sum_{n=1}^{N}\assign_{nq}\left(\compcov_q^{-1}(\sourcevec_n-\compmeanvec_q)(\sourcevec_n-\compmeanvec_q)\T\compcov_q^{-1}-\compcov^{-1}_q\right)
\end{equation}
Taking the expectation and setting to $0$, we can then isolate the covariance matrix, 
\begin{equation}
\compcov_q=\frac{1}{\sum_{n=1}^{N}\E{\assign_{nq}}}\left(\sum_{n=1}^{N}\E{\assign_{nq}}\E{(\sourcevec_n-\compmeanvec_q)(\sourcevec_n-\compmeanvec_q)\T|\assign_{nq}=1}\right)
\end{equation}
If we impose a conjugate inverse Wishart prior with log-density,
\begin{equation}
\ln \mathcal{W}^{-1}(\compcov|\bm{\Psi},\nu)=-\frac{\nu+p+1}{2}\ln|\compcov|-\frac{1}{2}\trace({\bm{\Psi}\compcov^{-1}})+\const
\end{equation}
we can add the derivative given by,
\begin{equation}
\nabla_{\compcov}\ln p(\compcov)=-\frac{\nu+p+1}{2}\compcov^{-1}+\frac{1}{2}\compcov^{-1}\bm{\Psi}\compcov^{-1}
\end{equation}
to compute the following MAP update as well,
\begin{equation}
\compcov_q=\frac{1}{\sum_{n=1}^{N}\E{\assign_{nq}}+\nu+p+1}\left(\Psi+\sum_{n=1}^{N}\E{\assign_{nq}}\E{(\sourcevec_n-\compmeanvec_q)(\sourcevec_n-\compmeanvec_q)\T|\assign_{nq}=1}\right).
\end{equation}

\subsection{Noise Variance}
\begin{equation}
\nabla_{\noisecov}\ln p(\obsvec,\sourcevec,\assignvec)=\frac{1}{2}\sum_{n=1}^{N}\left(\noisecov^{-1}(\obsvec_n-\mix\sourcevec_n)(\obsvec_n-\mix\sourcevec_n)\T\noisecov^{-1}-\noisecov^{-1}\right)
\end{equation}
Taking the diagonal elements $\noisescalesq_k$
\begin{equation}
\pdv{\noisescalesq_{k}}\ln p(\obsvec,\sourcevec,\assignvec)=\frac{1}{2}\left(\sum_{n=1}^{N}\left(\frac{\obs_{nk}-\bm{e}_k\T\mix\sourcevec_n}{\noisescalesq_{k}}\right)^2-N\frac{1}{\noisescalesq_k}\right)
\end{equation}
Taking the expectation and solving, we get
\begin{equation}
\noisescalesq_k=\frac{1}{N}\sum_{n=1}^{N}\E{(\obs_{nk}-\bm{e}_k\T\mix\sourcevec_n)^2}=\frac{1}{N}\sum_{n=1}^{N}(\obs_{nk}^2+\bm{e}_k\T\mix\E{\sourcevec_n\sourcevec_n\T}\mix\T\bm{e}_k-2\obs_{nk}\bm{e}_k\T\mix\E{\sourcevec_n})
\end{equation}
If $\noisescalesq_k$ is endowed with an inverse Gamma prior then it contributes the terms,
\begin{equation}
\nabla_{\noisescalesq}\ln p(\noisescalesq_k)=-(1+\alpha_k)\frac{1}{\noisescalesq_k}+\frac{\beta_k}{(\noisescalesq_k)^2}
\end{equation}
and adding those terms to the gradient we can solve again and find
\begin{equation*}
\noisescalesq_k=\frac{1}{N+2+2\alpha_k}\sum_{n=1}^{N}\left(2\beta_k+\E{(\obs_{nk}-\bm{e}_k\T\mix\sourcevec_n)^2}\right)
\end{equation*}
If we have a single $\noisescalesq_0$ controlling the noise level (scaled unit diagonal covariance), then the gradient simplifies
\begin{equation}
\pdv{\noisescalesq_{0}}\ln p(\obsvec,\sourcevec,\assignvec)=\frac{1}{(\noisescalesq_0)^2}\left(\frac{1}{2}\sum_{n=1}^{N}(\obsvec_n-\mix\sourcevec_n)\T(\obsvec_n-\mix\sourcevec_n)\right)-\frac{ND}{2}\frac{1}{\noisescalesq_0}
%-\frac{1}{2\noisescalesq_0}\left(\frac{1}{\noisescalesq_0}\sum_{n=1}^{N}\left(\sum_{k=1}^K\obs_{nk}-\Tr[\mix\E{\sourcevec_n\sourcevec_n\T}\mix\T]-2\obsvec_n\T\mix\E{\sourcevec_n}\right)-NK\right)
\end{equation}
and taking the expectation and isolating yields,
\begin{equation}
\noisescalesq_0=\frac{\sum_{n=1}^{N}(\obsvec_n\T\obsvec_n+\Trace(\mix\T\mix\E{\sourcevec_n\sourcevec_n\T})-2\obsvec_n\T\mix\E{\sourcevec_n})}{ND}
\end{equation}
or if $\noisecov$ is a scaled unit matrix and 
\begin{equation}
\noisescalesq_0=\frac{2\beta_0+\sum_{n=1}^{N}(\obsvec_n\T\obsvec_n+\Trace(\mix\T\mix\E{\sourcevec_n\sourcevec_n\T})-2\obsvec_n\T\mix\E{\sourcevec_n})}{ND+2+2\alpha_0}
\end{equation}

\subsection{Factor Loadings}
\begin{equation}
\nabla_{\mix}\ln p(\obsvec,\sourcevec,\assignvec)=\noisecov^{-1}\sum_{n=1}^{N}(\obsvec_n-\mix\sourcevec_n)\sourcevec_n\T
\end{equation}
we can then again take the expectation, 
\begin{equation}
\E{\nabla_{\mix}\ln p(\obsvec,\sourcevec,\assignvec)}=\noisecov^{-1}\left(\sum_{n=1}^{N}\obsvec_n\E{\sourcevec_n}\T-\mix\left(\sum_{n=1}^{N}\E{\sourcevec_n\sourcevec_n\T}\right)\right)
\end{equation}
and set the gradient to zero to find the optimal update
\begin{equation*}
\mix = \left(\sum_{n=1}^{N}\obsvec_n\E{\sourcevec_n}\T\right)\left(\sum_{n=1}^{N}\E{\sourcevec_n\sourcevec_n\T}\right)^{-1}
\end{equation*}
If we add a Gaussian prior $[\mix]_{ij}\sim \gaussrnd(0,\scalesq_{\mix})$ it contributes the term,
\begin{equation*}
\E{\nabla_{\mix}\ln p(\mix)}=-\frac{1}{\scalesq_{\mix}} \mix
\end{equation*}
and we can solve again to find the MAP update,
\begin{equation*}
\mix = \left(\sum_{n=1}^{N}\obsvec_n\E{\sourcevec_n}\T\right)\left(\sum_{n=1}^{N}\E{\sourcevec_n\sourcevec_n\T}+\frac{\noisescalesq}{\scalesq_{\mix}}\right)^{-1}
\end{equation*}





\end{document}